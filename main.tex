\documentclass[blue]{pastelcv}
% available options are: turquoise, red, lighthipsterblue, rose, blue, pastelgreen,
% grey, allblack
\usepackage[utf8x]{inputenc}

\usepackage[familydefault,light]{Chivo} %% Option 'familydefault' only if the base font of the document is to be sans serif
\usepackage[default]{raleway}
\usepackage[T1]{fontenc}

\getpagesetup

%------------------------------------------------------------------
\title{Pastel CV}
\author{\LaTeX{} Ninja}
\date{December 2019}


\begin{document}

\thispagestyle{empty}
%-------------------------------------------------------------

\begin{center}

\headername{Franklin A. R. Coêlho}{}

\jobdescription{Graduando em Engenharia de Computação} %~•~ Análise de Dados ~•~ Inteligência Artificial%}

%\adressline{The Black Pearl, Tortuga, The East Indies. The Black Pearl, Tortuga, The East Indies.}
 
\getgreyishblackfont

{
\small
\headersymbols{%
 \faPhone/83 98807 2528,
 \faAt/franklinanthony@eng.ci.ufpb.br,
 \faGithub/\href{https://github.com/franklinthony}{franklinthony}
}
}

%\headerquote{Make something of yourself instead of trying to impress people.}

\end{center}

\vspace{1cm}

%------------------------------------------------

\setupparacol
\begin{paracol}{2}


\fancysection{cvcolour}{}{Formação Acadêmica}
\begin{tabular}{r| p{\onethirdwidth}}
    \cvevent{2018}{Graduação em andamento em Engenharia de Computação}{Centro de Informática}{UFPB\color{cvaltcolour}}{}
    %\cvevent{2016--2017}{Captain of the Black Pearl}{Lead}{Tortuga \color{cvaltcolour}}{Found a secret treasure, lost the ship. \lorem}
\end{tabular}
\vspace{1em}

\fancysection{cvcolour}{}{Formação Complementar}

\begin{tabular}{r| p{\onethirdwidth}}
    \cvevent{2020}{TensorFlow 2.0 e Flask API: um guia completo}{UDEMY}{USA\color{cvaltcolour}}{} \\
    \cvevent{2020}{Deep Learning com Python}{UDEMY}{USA\color{cvaltcolour}}{} \\
    \cvevent{2019}{Machine Learning e Data Science com Python}{UDEMY}{USA\color{cvaltcolour}}{} \\
    \cvevent{2019}{Visão Computacional e suas Aplicações}{Centro de Informática}{UFPB\color{cvaltcolour}}{} \\
    \cvevent{2019}{Introdução ao TensorFlow}{Centro de Informática}{UFPB\color{cvaltcolour}}{} \\
    \cvevent{2018}{Ferramentas de Computação Interativa}{Centro de Informática}{UFPB\color{cvaltcolour}}{} \\
    \cvevent{2017}{Introdução a Python 3}{Centro de Informática}{UFPB\color{cvaltcolour}}{}
\end{tabular}
\vspace{1em}

\fancysection{cvcolour}{}{Programação}

\switchcolumn

\progressarc{2mm}{cvcolour}{1cm}{cvaltcolour}{\large\bf \textbf{C}}{Programação \\ Estruturada}{below}{80}
\progressarc{2mm}{cvcolour}{1cm}{cvaltcolour}{\large\bf \textbf{C++}}{Programação \\ Orientada a Objeto}{below}{80}
\progressarc{2mm}{cvcolour}{1cm}{cvaltcolour}{\large\bf \textbf{Python}}{Inteligência \\ Artificial}{below}{70}
\vspace{1em}

%fancysection{cvcolour}{}{Programação \#2}
%
%\begin{minipage}[t]{\paracolwidth}
%\begin{tabular}{r @{\hspace{0.5em}}l}
%     \LaTeX{} & \barrule{0.6}{0.5em}{cvcolour} \\
%     Markdown & \barrule{0.25}{0.5em}{cvcolour} \\
%     javascript & \barrule{0.1}{0.5em}{cvcolour} \\
%\end{tabular}
%\end{minipage}
%\vspace{1em}
%

%-----------------------------------------------------------

\fancysection{cvcolour}{}{Idiomas}

\begin{tabular}{l | ll}
\textbf{Português} & {\phantom{x}\footnotesize Língua materna} \\
\textbf{Inglês} & \pictofraction{\faCircle}{cvcolour}{2}{black!30}{3}{\tiny} \\
\textbf{Espanhol} & \pictofraction{\faCircle}{cvcolour}{2}{black!30}{3}{\tiny}
\end{tabular}
\vspace{1em}

\fancysection{cvcolour}{}{Atuação}

\textsc{\large Bolsista}\\[0.5em]

\begin{tabular}{r| p{\onethirdwidth}}
    \cvevent{2020}{Aluno extensionista no projeto "Comissão de Gestão Ambiental - Gerenciamento de Resíduos Eletroeletrônicos"}{Centro de Informática}{UFPB\color{cvaltcolour}}{} \\
    \cvevent{2019}{Tutor da disciplina Cálculo Diferencial e Integral II}{Centro de Informática}{UFPB\color{cvaltcolour}}{Programa de Tutoria (PROTUT), do Centro de Informática da Universidade Federal da Paraíba} \\
    \cvevent{2018}{Tutor da disciplina Cálculo Vetorial e Geometria Analítica}{Centro de Informática}{UFPB\color{cvaltcolour}}{Programa de Tutoria (PROTUT), do Centro de Informática da Universidade Federal da Paraíba}
\end{tabular}
\vspace{1em}

\textsc{\large Voluntário}\\[0.5em]

\begin{tabular}{r| p{\onethirdwidth}}
    \cvevent{2019}{Aluno extensionista no projeto "Multiverso STEM: difusão do conhecimento em ciência, tecnologia, matemática e computação"}{Centro de Informática}{UFPB\color{cvaltcolour}}{}
\end{tabular}
\vspace{1em}

%-----------------------------------------------------------
\switchcolumn
    
\begin{tabular}{r| p{\onethirdwidth}}
    \cvevent{2019}{Aluno extensionista  na Equipe Literacia de Inovação Tecnológica em Saúde -- ELITS}{Centro de Ciências da Saúde}{UFPB\color{cvaltcolour}}{}\\
    \cvevent{2018}{Aluno extensionista no projeto "www.BMC: mostrando a Matemática Computacional da UFPB para o mundo através da internet"}{Centro de Informática}{UFPB\color{cvaltcolour}}{}\\
    \cvevent{2018}{Aluno extensionista no projeto "Pensamento Computacional no auxílio das aulas de Robótica na EEEFM José Lins do Rego -- Pilar -- PB"}{Centro de Informática}{UFPB\color{cvaltcolour}}{}\\
  \cvevent{2017}{Secretário júnior na instituição Compassion International}{Campina Grande}{PB\color{cvaltcolour}}{A Compassion International é uma organização evangélica interdenominacional, sem fins lucrativos, cujo objetivo é ajudar crianças em situação de risco ao redor do mundo}
\end{tabular}
\vspace{1em}

\fancysection{cvcolour}{}{Patentes \& Registros}

\begin{tabular}{r| p{\onethirdwidth}}
    \cvevent{2019}{Otoscópio Pediátrico com Captação de Imagem}{Universidade Federal da Paraíba}{UFPB\color{cvaltcolour}}{\textbf{Instituição de registro}: INPI - Instituto Nacional da Propriedade Industrial \newline \textbf{Número do registro}: BR10201902817}
\end{tabular}
\vspace{1em}

\fancysection{cvcolour}{}{Prêmios \& Títulos}

\begin{tabular}{r| p{\onethirdwidth}}
    \cvevent{2019}{Prêmio de melhor trabalho no Encontro de Iniciação à Docência}{Universidade Federal da Paraíba}{UFPB\color{cvaltcolour}}{Resumo "E-books Tutoriais para Cursos Introdutórios de Matemática: Algoritmos
Computacionais no Suporte à Resolução de Problemas" (PROTUT/CI 2019)}
\end{tabular}
\vspace{1em}

\switchcolumn

\begin{tabular}{r| p{\onethirdwidth}}
    \cvevent{2018}{Prêmio Elo Cidadão de Extensão Universitária}{Universidade Federal da Paraíba}{UFPB\color{cvaltcolour}}{Projeto "Pensamento Computacional no auxílio das aulas de Robótica na EEEFM José Lins do Rego -- Pilar -- PB"}
\end{tabular}
\vspace{1em}

\fancysection{cvcolour}{}{Produções}

\textsc{\large Resumos publicados}\\[0.5em]

\begin{tabular}{r| p{\onethirdwidth}}
    \cvevent{2019}{Desenvolvimento de E-books Estáticos e Interativos sobre Geometria Analítica e Cálculo Vetorial para E-tutoring em Ciências Computacionais e Engenharia}{Fortaleza}{CE\color{cvaltcolour}}{XLVII COBENGE - Congresso Brasileiro de Educação em Engenharia}\\
    \cvevent{2018}{Tendências de e-Tutoring em Cálculo Vetorial e Geometria Analítica Para Ciências Matemáticas e Computacionais: Estudo de Caso do PROTUT/UFPB}{Recife}{PE\color{cvaltcolour}}{V CONEDU - Congresso Nacional de Educação}
\end{tabular}
\vspace{1em}

\end{paracol}

\end{document}
